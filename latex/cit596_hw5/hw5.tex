\documentclass{article}

\usepackage{amsmath,amssymb,amsthm,graphicx,subfigure,qtree}

\pagestyle{myheadings}

\pdfpagewidth 8.5in
\pdfpageheight 11 in

\setlength\topmargin{0in}
\setlength\textheight{8.5in}
\setlength\textwidth{6.5in}
\setlength\oddsidemargin{0in}
\setlength\evensidemargin{0in}

\newcommand{\suchthat}{\ni}
\newcommand{\onlyif}{\Longleftrighttriangle}
\newcommand{\definedby}{\triangleq}
\newcommand{\union}{\bigcup}
\newcommand{\intersect}{\bigcap}
\newcommand{\where}{\mid}
\newcommand{\inverse}{\overline}

\title{CIT 596 Homework 5}
\author{Steven Tomcavage\\stomcava@seas.upenn.edu}
\date{March 23, 2011}

\markboth{\hfill Steven Tomcavage }{\hfill Steven Tomcavage }

\begin{document}

\maketitle

\section{Exercise 2.31}

Let $B$ be the language of all palindromes over $\{0, 1\}$ containing an equal
number of $0$s and $1$s. Show that $B$ is not context free.

TODO

\section{Exercise 2.36}

Give and example of a language that is not context free but that acts like a CFL
in the pumping lemma. Prove that your example works. 

TODO

\section{Exercise 1.54}

Consider the language $F = \{a^i b^j c^k \where i, j, k \geq 0 \text{ and if }
i = 1 \text{ then } j = k\}$. 

\subsection{Part a}
Show that $F$ is not regular.

TODO

\subsection{Part b}

Show that $F$ acts like a regular language in the pumping lemma. In other words,
give a pumping length $p$ and show that $F$ satisfies the three conditions of
the pumping lemma for this value of $p$.

TODO

\subsection{Part c}

Explain why parts a and b do not contradict the pumping lemma.

TODO

\section{Exercise 2.40}

Say that a language is prefix-closed if the prefix of any string in the language
is also in the language. Let $C$ be an infinite, prefix-closed, context-free
language. Show that $C$ contains an infinite regular subset.

TODO 

\section{Exercise 2.44}

If $A$ and $B$ are languages, define $A \diamond B = \{xy \where x \in A \text{
and } y \in B \text{ and } |x| = |y|\}$. Show that if $A$ and $B$ are regular
languages, then $A \diamond B$ is a CFL.

TODO 

\section{Exercise 3.1d}

TODO 

\section{Exercise 3.2e}

TODO 

\section{Exercise 3.9}

TODO 

\section{Exercise 3.13}

TODO 

\section{Exercise 3.15}

TODO 

\section{Exercise 3.16b}

TODO 

\section{Exercise 4.15}

TODO 

\section{Exercise 4.16}

TODO 

\section{Exercise 4.19}

TODO 

\section{Exercise 3.3}

TODO -- see note in homework assignment

\section{Exercise 2.20a}

TODO -- see note in homework assignment

\section{Exercise 2.20b}

TODO -- see note in homework assignment

\end{document}
