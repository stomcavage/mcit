\documentclass{article}

\usepackage{amsmath,amssymb,amsthm,graphicx,subfigure,qtree}

\pagestyle{myheadings}

\pdfpagewidth 8.5in
\pdfpageheight 11 in

\setlength\topmargin{0in}
\setlength\textheight{8.5in}
\setlength\textwidth{6.5in}
\setlength\oddsidemargin{0in}
\setlength\evensidemargin{0in}

\newcommand{\suchthat}{\ni}
\newcommand{\onlyif}{\Longleftrighttriangle}
\newcommand{\definedby}{\triangleq}
\newcommand{\union}{\bigcup}
\newcommand{\intersect}{\bigcap}
\newcommand{\where}{\mid}
\newcommand{\inverse}{\overline}

\title{CIT 596 Homework 4}
\author{Steven Tomcavage\\stomcava@seas.upenn.edu}
\date{March 3, 2011}

\markboth{\hfill Steven Tomcavage }{\hfill Steven Tomcavage }

\begin{document}

\maketitle

\section{Exercise 2.1}

\subsection{Part a}

\Tree [.E [.T [.F [.a ] ] ] ]

\subsection{Part b}

\Tree [.E [ [.E [.T [.F [.a ] ] ] ] [ + ] [.T [.F [.a ] ] ] ] ]

\subsection{Part c}

\Tree [.E [ [.E [ [.E [.T [.F [.a ] ] ] ] [ + ] [.T [.F [.a ] ] ] ] ] [ + ] [.T [.F [.a ] ] ] ] ]

\subsection{Part d}

\Tree [.E [.T [.F [ ( ] [.E [.T [.F [ ( ] [.E [.T [.F [.a ] ] ] ] [ ) ] ] ] ] [
) ] ] ] ]

\section{Exercise 2.2}

TODO

\section{Exercise 2.4b}

Given $\Sigma = \{0, 1\}$, give a CFG that generates the language $\{ w \where
w \text{ starts and ends with the same symbol} \}$.

\begin{align*}
	&S \rightarrow 0A0 \mid 1A1\\
	&A \rightarrow 0 \mid 1 \mid A \mid \epsilon
\end{align*}

\section{Exercise 2.4c}

Given $\Sigma = \{0, 1\}$, give a CFG that generates the language $\{ w \where
w \text{ the length of } w \text{ is odd} \}$.

\begin{align*}
	&S \rightarrow 0A \mid 1A\\
	&A \rightarrow 00 \mid 01 \mid 10 \mid 11 \mid A \mid \epsilon
\end{align*}

\section{Exercise 2.4e}

Given $\Sigma = \{0, 1\}$, give a CFG that generates the language $\{ w \where
w = w^R \text{, that is} w \text{is a palindrome} \}$.

\begin{align*}
	&S \rightarrow 0A0 \mid 1A1 \mid \epsilon\\
	&A \rightarrow S
\end{align*}

\section{Exercise 2.5b}

Give a formal description and state diagram for the language describe by
Exercise 2.4b.

TODO

\section{Exercise 2.5c}

Give a formal description and state diagram for the language describe by
Exercise 2.4c.

TODO

\section{Exercise 2.5e}

Give a formal description and state diagram for the language describe by
Exercise 2.4e.

TODO

\section{Exercise 2.9}

Give a CFG that generates the langage $A = \{ a^i b^j c^k \where i = j \text{
or } j = k \text{ where } i, j, k \geq 0 \}$. Is this CFG ambiguous?

TODO

\section{Exercise 2.13}

\subsection{Part a}

$L(G)$ generates a string of zeros with one or two hash marks in the string.
If there are two hash marks in the string, the hash marks can be at the
beginning, end, or anywhere in the middle. If there is only one hash mark, the
number of zeros after the hash mark is twice the number of zeros before the hash mark.

\subsection{Part b}

TODO

\section{Exercise 2.14}

Convert the following CFG to Chomsky Normal Form:

\begin{align*}
	& A \rightarrow BAB \mid B \mid \epsilon\\
	& B \rightarrow 00 \mid \epsilon
\end{align*}

\subsection*{Step 1}
	Add a new start variable
	\begin{align*}
		& S \rightarrow A\\
		& A \rightarrow BAB \mid B \mid \epsilon\\
		& B \rightarrow 00 \mid \epsilon
	\end{align*}
\subsection*{Step 2}
	Eliminate $B \rightarrow \epsilon$
	\begin{align*}
		& S \rightarrow A\\
		& A \rightarrow BAB \mid B \mid \epsilon \mid BA \mid AB\\
		& B \rightarrow 00 
	\end{align*}
\subsection*{Step 3}
  Eliminate $A \rightarrow \epsilon$
	\begin{align*}
		& S \rightarrow A \mid \epsilon\\
		& A \rightarrow BAB \mid B \mid BA \mid AB \mid BB\\
		& B \rightarrow 00 
	\end{align*}
\subsection*{Step 4}
  Remove unit rules
	\begin{align*}
		& S \rightarrow BAB \mid 00 \mid BA \mid AB \mid BB \mid \epsilon \\
		& A \rightarrow BAB \mid 00 \mid BA \mid AB \mid BB\\
		& B \rightarrow 00 
	\end{align*}

\section{Exercise 2.20}

Let $A/B = \{ w | wx \in A \text{ for some } x \text{ in } B$. Show that, if $A$
is context free and $B$ is regular, than $A/B$ is context free.

\begin{proof}
	\mbox{}
	\begin{enumerate}
		\item Given that $A/B$ is context free.
		\item Given that $wx \in A$ for some $x \in B$.
		\item Then $B$ is contained in the language $A$.
		\item Therefore, $B$ is context free. \qedhere
	\end{enumerate}
\end{proof}

\section{Exercise 2.26}

Show that, if $G$ is a CFG in Chomsky normal form, then for any string $w \in
L(G)$ of length $n \geq 1$, exactly $2n - 1$ steps are required for any
derivation of $w$.

\begin{proof}
	\mbox{}
	\begin{enumerate}
	  \item Given that CNF only allows productions of the form $A \rightarrow BC$
	  and $A \rightarrow a$.
	  \item If $n = 1$, then the only production required to generate $w$
	  is of the form $A \rightarrow a$, thus the length is $1 = 2n - 1$.
	  \item If $n = 2$, then three productions are required, one of
	  the form $A \rightarrow BC$ and two of the form $A \rightarrow a$, thus the
	  number of steps in the derivation is $3 = 2n - 1$.
	  \item If $n > 2$, then we can break $w$ down into units of size 2 or 1 and
	  combine those units to form the larger derivation, which will have $(2n_i - 1) +
	  (2n_j - 1) + \ldots + (2n - 1)$ steps.
	  \item Therefore, the number of steps required to derive $w$ is $2n - 1$.
	  \qedhere
	\end{enumerate}
\end{proof}

\section{Exercise 2.30a}

Use the pumping lemma to show the langauge $\{ 0^n 1^n 0^n 1^n \where n \geq 0
\}$ is not context free.

TODO

\end{document}
