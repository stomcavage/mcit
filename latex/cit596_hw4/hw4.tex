\documentclass{article}

\usepackage{amsmath,amssymb,amsthm,graphicx,subfigure,qtree}

\pagestyle{myheadings}

\pdfpagewidth 8.5in
\pdfpageheight 11 in

\setlength\topmargin{0in}
\setlength\textheight{8.5in}
\setlength\textwidth{6.5in}
\setlength\oddsidemargin{0in}
\setlength\evensidemargin{0in}

\newcommand{\suchthat}{\ni}
\newcommand{\onlyif}{\Longleftrighttriangle}
\newcommand{\definedby}{\triangleq}
\newcommand{\union}{\bigcup}
\newcommand{\intersect}{\bigcap}
\newcommand{\where}{\mid}
\newcommand{\inverse}{\overline}

\title{CIT 596 Homework 4}
\author{Steven Tomcavage\\stomcava@seas.upenn.edu}
\date{March 3, 2011}

\markboth{\hfill Steven Tomcavage }{\hfill Steven Tomcavage }

\begin{document}

\maketitle

\section{Exercise 2.1}

Given the following CFG, provide parse trees for the string in each part:

\begin{align*}
	& E \rightarrow E + T \mid T\\
	& T \rightarrow T \times F \mid F\\
	& F \rightarrow (E) \mid a
\end{align*}

\subsection{Part a: $a$}

\Tree [.E [.T [.F [.a ] ] ] ]

\subsection{Part b: $a + a$}

\Tree [.E [ [.E [.T [.F [.a ] ] ] ] [ + ] [.T [.F [.a ] ] ] ] ]

\subsection{Part c: $a + a + a$}

\Tree [.E [ [.E [ [.E [.T [.F [.a ] ] ] ] [ + ] [.T [.F [.a ] ] ] ] ] [ + ] [.T [.F [.a ] ] ] ] ]

\subsection{Part d : $((a))$}

\Tree [.E [.T [.F [ ( ] [.E [.T [.F [ ( ] [.E [.T [.F [.a ] ] ] ] [ ) ] ] ] ] [
) ] ] ] ]

\section{Exercise 2.2}

\subsection{Part a}

Use the languages $A = \{ a^m b^n c^n \where m, n \geq 0 \}$ and $B = \{ a^n b^n
c^m \where m, n \geq 0 \}$ together with Example 2.36 from Sipser to show the
class of context-free languages is not closed under intersection.

\begin{proof}
	\mbox{}
	\begin{enumerate}
	  \item Assume the class of context-free languages is closed under
	  intersection.
	  \item Given that $A$ and $B$ are context-free languages.
	  \item The intersection of $A$ and $B$ is the language $C = \{ a^i b^i c^i 
	  \where i \geq 0\}$. 
	  \item By Example 2.36, $C$ is not a context-free language.
	  \item Thus, the intersection of $A$ and $B$ is not a context-free language.
	  \item Therefore, the class of context-free languages is not closed under
	  intersection. \qedhere
	\end{enumerate}
\end{proof}

\subsection{Part b}

Use Part (a) and DeMorgan's law to show that the class of context-free languages
is not closed under complementation.

\begin{proof}
	\mbox{}
	\begin{enumerate}
	  \item TODO
	\end{enumerate}
\end{proof}

\section{Exercise 2.4b}

Given $\Sigma = \{0, 1\}$, give a CFG that generates the language $\{ w \where
w \text{ starts and ends with the same symbol} \}$.

\begin{align*}
	&S \rightarrow 0A0 \mid 1A1\\
	&A \rightarrow 0 \mid 1 \mid A \mid \epsilon
\end{align*}

\section{Exercise 2.4c}

Given $\Sigma = \{0, 1\}$, give a CFG that generates the language $\{ w \where
w \text{ the length of } w \text{ is odd} \}$.

\begin{align*}
	&S \rightarrow 0A \mid 1A\\
	&A \rightarrow 00 \mid 01 \mid 10 \mid 11 \mid A \mid \epsilon
\end{align*}

\section{Exercise 2.4e}

Given $\Sigma = \{0, 1\}$, give a CFG that generates the language $\{ w \where
w = w^R \text{, that is} w \text{is a palindrome} \}$.

\begin{align*}
	&S \rightarrow 0A0 \mid 1A1 \mid \epsilon\\
	&A \rightarrow S
\end{align*}

\section{Exercise 2.5b}

Give an informal description and state diagram for the language describe by
Exercise 2.4b.

\begin{figure}[h!]
	\centering
	\includegraphics[height=1.0in]{2_5_b.png}
	\caption{PDA for Exercise 2.5b}
\end{figure}

\section{Exercise 2.5c}

Give an informal description and state diagram for the language describe by
Exercise 2.4c.

\begin{figure}[h!]
	\centering
	\includegraphics[height=2.0in]{2_5_c.png}
	\caption{PDA for Exercise 2.5c}
\end{figure}

\section{Exercise 2.5e}

Give an informal description and state diagram for the language describe by
Exercise 2.4e.

\begin{figure}[h!]
	\centering
	\includegraphics[height=1.5in]{2_5_e.png}
	\caption{PDA for Exercise 2.5e}
\end{figure}

\section{Exercise 2.9}

Give a CFG that generates the langage $A = \{ a^i b^j c^k \where i = j \text{
or } j = k \text{ where } i, j, k \geq 0 \}$. Is this CFG ambiguous?

\subsection{Part a}

\begin{align*}
	& S \rightarrow Wc \mid aX\\
	& W \rightarrow aWbY\\
	& X \rightarrow bXcZ\\
	& Y \rightarrow W \mid \epsilon\\
	& Z \rightarrow X \mid \epsilon
\end{align*}

\subsection{Part b}

No, this CFG is not ambiguous because the leftmost derivation of any string
only generates one parse tree. The productions progress linearally and the only
loops in the CFG always loop back to the same location. There is never an option
for a loop to have a choice of where to return to.

\section{Exercise 2.13}

\subsection{Part a}

$L(G)$ generates a string of zeros with one or two hash marks in the string.
If there are two hash marks in the string, the hash marks can be at the
beginning, end, or anywhere in the middle. If there is only one hash mark, the
number of zeros after the hash mark is twice the number of zeros before the hash mark.

\subsection{Part b}

$L(G)$ is not regular because it contains two recursive rules. $T$ is formed
from either a hash mark or from some combination of 0 and $T$. Likewise, $U$ is
formed from either a hash mark or from some combination of 0 and $U$. These
structures cannot be described by regular expressions. Since regular languages
are described by regular expressions, the language $L(G)$ is not regular.

\section{Exercise 2.14}

Convert the following CFG to Chomsky Normal Form:

\begin{align*}
	& A \rightarrow BAB \mid B \mid \epsilon\\
	& B \rightarrow 00 \mid \epsilon
\end{align*}

\subsection*{Step 1}
	Add a new start variable
	\begin{align*}
		& S \rightarrow A\\
		& A \rightarrow BAB \mid B \mid \epsilon\\
		& B \rightarrow 00 \mid \epsilon
	\end{align*}
\subsection*{Step 2}
	Eliminate $B \rightarrow \epsilon$
	\begin{align*}
		& S \rightarrow A\\
		& A \rightarrow BAB \mid B \mid \epsilon \mid BA \mid AB\\
		& B \rightarrow 00 
	\end{align*}
\subsection*{Step 3}
  Eliminate $A \rightarrow \epsilon$
	\begin{align*}
		& S \rightarrow A \mid \epsilon\\
		& A \rightarrow BAB \mid B \mid BA \mid AB \mid BB\\
		& B \rightarrow 00 
	\end{align*}
\subsection*{Step 4}
  Remove unit rules
	\begin{align*}
		& S \rightarrow BAB \mid 00 \mid BA \mid AB \mid BB \mid \epsilon \\
		& A \rightarrow BAB \mid 00 \mid BA \mid AB \mid BB\\
		& B \rightarrow 00 
	\end{align*}

\section{Exercise 2.20}

Let $A/B = \{ w | wx \in A \text{ for some } x \text{ in } B$. Show that, if $A$
is context-free and $B$ is regular, than $A/B$ is context-free.

\begin{proof}
	\mbox{}
	\begin{enumerate}
		\item Given that $A/B$ is context-free.
		\item Given that $wx \in A$ for some $x \in B$.
		\item Then $B$ is contained in the language $A$.
		\item Therefore, $B$ is context-free. \qedhere
	\end{enumerate}
\end{proof}

\section{Exercise 2.26}

Show that, if $G$ is a CFG in Chomsky normal form, then for any string $w \in
L(G)$ of length $n \geq 1$, exactly $2n - 1$ steps are required for any
derivation of $w$.

\begin{proof}
	\mbox{}
	\begin{enumerate}
	  \item Given that CNF only allows productions of the form $A \rightarrow BC$
	  and $A \rightarrow a$.
	  \item If $n = 1$, then the only production required to generate $w$
	  is of the form $A \rightarrow a$, thus the length is $1 = 2n - 1$.
	  \item If $n = 2$, then three productions are required, one of
	  the form $A \rightarrow BC$ and two of the form $A \rightarrow a$, thus the
	  number of steps in the derivation is $3 = 2n - 1$.
	  \item If $n > 2$, then we can break $w$ down into units of size 2 or 1 and
	  combine those units to form the larger derivation, which will have $(2n_i - 1) +
	  (2n_j - 1) + \ldots + (2n - 1)$ steps.
	  \item Therefore, the number of steps required to derive $w$ is $2n - 1$.
	  \qedhere
	\end{enumerate}
\end{proof}

\end{document}
