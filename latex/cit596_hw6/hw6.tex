\documentclass{article}

\usepackage{amsmath,amssymb,amsthm,graphicx,subfigure,qtree,verbatim}

\pagestyle{myheadings}

\pdfpagewidth 8.5in
\pdfpageheight 11 in
 
\setlength\topmargin{0in}
\setlength\textheight{8.5in}
\setlength\textwidth{6.5in}
\setlength\oddsidemargin{0in}
\setlength\evensidemargin{0in}

\newcommand{\suchthat}{\ni}
\newcommand{\onlyif}{\Longleftrighttriangle}
\newcommand{\definedby}{\triangleq}
\newcommand{\union}{\bigcup}
\newcommand{\intersect}{\bigcap}
\newcommand{\where}{\mid}
\newcommand{\inverse}{\overline}

\title{CIT 596 Homework 6}
\author{Steven Tomcavage\\stomcava@seas.upenn.edu}
\date{April 17, 2011}

\markboth{\hfill Steven Tomcavage }{\hfill Steven Tomcavage }

\begin{document}

\maketitle

\section{Exercise 5.1}

Show that $EQ_{CFG}$ is undecidable.

\begin{proof}
	\mbox{}
	\begin{enumerate}
	  \item Assume $EQ_{CFG}$ is decidable. 
	  \item Let $S$ be a CFG and $T$ be a CFG that accepts all possible strings
	  from the tokens used by $S$.
	  \item To determine if $S = T$ requires deciding $ALL_{CFG}$, where
	  $ALL_{CFG}$ accepts when $S$ generates all possible strings from its
	  tokens.
	  \item By Theorem 5.13 in Sipser, $ALL_{CFG}$ is undecidable, and therefore,
	  $EQ_{CFG}$ is undecidable. \qedhere
	\end{enumerate}
\end{proof}

\section{Exercise 5.3}

Find a match in the following instance of the Post Correspondance Problem.

\begin{equation*}
	\left\{ \left[\frac{ab}{abab}\right], \left[\frac{b}{a}\right],
	\left[\frac{aba}{b}\right], \left[\frac{aa}{a}\right] \right\}
\end{equation*}

Here a Prolog program to find a solution to a given Post Correspondance problem:

\verbatiminput{postCorrespondance.pl}

Running this program under SWI-Prolog on the given problem yields the following
solutions:
 
\begin{verbatim}
pdt_reload('c:/dropbox/usr/mcit/prolog/postcorrespondance.pl').
|    c:/dropbox/usr/mcit/prolog/postcorrespondance.pl
% c:/dropbox/usr/mcit/prolog/postcorrespondance compiled 0.00 sec, 2,872 bytes
true.

17 ?- solve([("ab", "abab"), ("b", "a"), ("aba", "b"), ("aa", "a")], Solution).
Solution = [4, 4, 2, 1] ;
;
Solution = [1, 1, 3, 2, 2, 4, 4] .
\end{verbatim}

So the solutions are:

\begin{equation*}
	\left\{ 
		\left[\frac{aa}{a}\right], 
		\left[\frac{aa}{a}\right],
		\left[\frac{b}{a}\right], 
		\left[\frac{ab}{abab}\right] 
	\right\}
\end{equation*}

\begin{equation*}
	\left\{ 
		\left[\frac{ab}{abab}\right], 
		\left[\frac{ab}{abab}\right],
		\left[\frac{aba}{b}\right], 
		\left[\frac{b}{a}\right], 
		\left[\frac{b}{a}\right] 
		\left[\frac{aa}{a}\right], 
		\left[\frac{aa}{a}\right]
	\right\}
\end{equation*}

\section{Exercise 5.4}

If $A \leq_m B$ and $B$ is a regular language, does that imply that $A$ is a
regular language? Why or why not?

No, it does not imply that $A$ is a regular language. $B$ is the set of strings
generated by $f(w)$ where $f$ is a function and $w$ is a string generated by
$A$. The mapping is not necessarily a one-to-one relationship between $w$ and
$f(w)$. Thus, there can exist a function, $f$, that maps from a non-regular language to
a regular language.

\section{Exercise 5.5}

Show that $A_{TM}$ is not mapping reducible to $E_{TM}$. 

\begin{proof}
	\mbox{}
	\begin{enumerate}
	  \item Assume that $A_{TM} \leq_m E_{TM}$.
	  \item Let $M$ be the TM for $\langle E_{TM}, w \rangle$.
	  \item Build a TM, $N$, so that on $w$, if $M$ rejects, then $A$ accepts, and
	  if $M$ accepts, then $A$ rejects.
	  \item But, because $E_{TM}$ is undecidable, then $A_{TM}$ must be
	  undecidable. 
	  \item Therefore, $A_{TM}$ is not mapping reducible to $E_{TM}$.\qedhere
	\end{enumerate}
\end{proof}

\section{Exercise 5.6}

Show that $leq_m$ is a transitive relation.

\begin{proof}
	\mbox{}
	\begin{enumerate}
	  \item Let $A \leq_m B$. This means there is a function $f_1$ that accepts a
	  $w$ in A and the output of $f_1$ is in $B$.
	  \item Let $B \leq_m C$. This means there is a function $f_2$ that accepts a
	  $w$ in B and the output of $f_2$ is in $C$.
	  \item Thus, since $B$ is a subset of $A$ and $C$ is a subset of $B$,
	  then $C$ must be a subset of $A$ and there must be a function $f_3$ that
	  accepts a $w$ in $A$ and the output of $f_3$ is in $C$.
	  \item Therefore, $leq_m$ is a transitive relation. \qedhere
	\end{enumerate}
\end{proof}

\section{Exercise 5.7}

Show that if $A$ is Turing-recognizable and $A \leq_m \not{A}$, then $A$ is
decidable.

\begin{proof}
	\mbox{}
	\begin{enumerate}
	  \item Let $B$ be the language $\not{A}$.
	  \item Assume that $B$ is Turing-recognizable.
	  \item Since $A$ is mapping reducable to $B$, then given a $w$ in the laguage
	  of $A$, if $B$ accepts $w$, $A$ will reject $w$, and if $B$ rejects $w$,
	  then $A$ will accept $w$.
	  \item Thus, $B$ is a co-Turing recognizer for $A$. 
	  \item Therefore, $A$ is Turing-recognizable and $B$ is co-Turing recognizable
	  for $A$, then $A$ is decidable. \qedhere
	\end{enumerate}
\end{proof}

\section{Exercise 5.9}

Let $T = \{ \langle M \rangle \where M \text{ is a TM that accepts } w^R \text{
whenever it accepts } w \}$. Show that $T$ is undecidable. 

This is a restatement of the halting problem. Since we've already proved that
the halting problem is undecidable, then this problem is undecidable.

\section{Exercise 5.17}

Show that the Post Correspondence problem is decidable over the unary alphabet
$\Sigma = \{ 1 \}$.

\begin{proof}
	\mbox{}
	\begin{enumerate}
	  \item Given that the Post Corrspondence problem requires each charcter in
	  the top and bottom strings to match and for the top and bottom strings to be
	  the same length.
	  \item Using a unary alphabet, each character on the top is guaranteed to
	  match its corresponding character on the bottom. 
	  \item Determining whether the items can be arranged so the top and bottom
	  lengths are the same is a matter of trying all possible combinations of
	  items without repeating items. If a combination is found where the length of
	  the top matches the length of the bottom, then the machine accepts. Otherwise
	  the machine rejects.
	  \item Therefore the Post Correspondence problem is decidable over the unary
	  alphabet. \qedhere
	\end{enumerate}
\end{proof}

\section{Exercise 5.19}

In the silly Post Correspondence Problem, SPCP, in each pair the top string has
the same length as the bottom string. Show that the SPCP is decidable.

\begin{proof}
	\mbox{}
	\begin{enumerate}
	  \item Given that the Post Corrspondence problem requires each charcter in
	  the top and bottom strings to match and for the top and bottom strings to be
	  the same length.
	  \item If all items have the same length between the top and bottom, then the
	  machine needs to search for any item where the characters on the top match
	  the characters on the bottom. If one item like that is found, then the
	  machine accepts. Otherwise the machine will never be able to build a correct
	  string, and it rejects.
	  \item Therefore the Silly Post Correspondence problem is decidable. \qedhere
	\end{enumerate}
\end{proof}

\section{Exercise 5.28}

Let $P$ be a language consisting of Turing machine descriptions where $P$
fulfills two conditions. First $P$ is nontrivial - it contains some, but not
all, TM descriptions. Second, $P$ is a property of the TM's language - whenever
$L(M_1) = L(M_2)$, we have $\langle M_1 \rangle \in P \text{ iif } \langle M_2
\rangle \in P$. Here, $M1$ and $M2$ are any TMs. Prove that $P$ is an
undecidable language.

TODO

\section{Exercise 5.29}

Show that both conditions in Exercise 5.28 are necessary for proving that $P$ is
undecidable.

TODO

\section{Exercise 6.1}

Give an example in the spirit of the recursion theorem of a program in a real
programming language that prints itself out.

Here is a program written in Clojure which prints itself out. To run it
(assuming the code is in a file named quine.clj and Clojure is installed on the
machine), start the Clojure REPL with the 
command \verb|java -cp clojure.jar clojure.main|. Then load and execute the file
with the command \verb|(load-file "./quine.clj")|.

\verbatiminput{quine.clj}

\section{Exercise 6.3}

Show that if $A \leq_T B$ and $B \leq_T C$ then $A \leq_T C$.

\begin{proof}
	\mbox{}
	\begin{enumerate}
	  \item The statement $A \leq_T B$ means that $A$ is a decider if there exists
	  some language $B$ that $A$ reduces to. 
	  \item Likewise, $B \leq_T C$ means that $B$ is a decider if there exists some
	  language $C$ that $B$ reduces to. 
	  \item Assuming from the first statement that a language $B$ exists, and from
	  the second statement that $B$ is a decider, then $A \leq_m B \leq_T C$.  
	  \item Since Exercise 5.6 shows that $\leq_m$ is a transitive relation,
	  then $A \leq_m B \leq_T C$ means that $A \leq_T C$. 
	  \item Therefore, if $A \leq_T B$ and $B \leq_T C$ then $A \leq_T C$. \qedhere
	\end{enumerate}
\end{proof}

\section{Exercise 6.5}

Is the statement $\exists x \forall y [ x + y = y]$ a member of $\text{Th}(N,
+)$? Why or why not? What about $\exists x \forall y [ x + y = x]$?

Assuming that $0 \in N$, then the first statement, $\exists x \forall y [ x + y
= y]$, is a member of $\text{Th}(N, +)$ because it is true when $x = 0$. So for
all $y$, $0 + y = y$.

The second statement, $\exists x \forall y [ x + y = x]$, is not a member of $\text{Th}(N,
+)$, because it's only true where $y = 0$. In that case, $x + 0 = x$, but in all
other cases in $N$, it is false. 

\section{Exercise 6.6}

Describe two different Turing machines, $M$ and $N$, that, when started on any
input, $M$ outputs $\langle N \rangle$ and $N$ outputs $\langle M \rangle$.

The machine $M$ contains the description of the machine $N$. For each
configuration in $M$, the machine reads a step from the description of $N$,
outputs that step, and moves right to the next step in the description of $N$. 

Likewise, the machine $N$ contains a description of $M$ and opterates on that
description in the same manner as $M$ opperates on the description it contains.

\section{Exercise 6.10}

Give a model of the sentence provided in the text (omitted).

The model is $(\{ \bf{C}, \bf{R} \}, =)$.

\section{Exercise 6.17}

Let $A$ and $B$ be two disjoint languages. Say that language $C$ seperates $A$
and $B$ if $A \subseteq C$ and $B \subseteq \not{C}$. Describe two
Turing-recognizable languages that aren't separated by any decidable language.

This problem is searching for a $C$ that is undecidable where $A \leq_T C$ and
$B \leq_T \not{C}$. I'm not sure where I can find such a $C$.

\section{Exercise 6.25}

Show that for any $c$, some strings $x$ and $y$ exist where $K(xy) > K(x) +
K(y) + c$. 

To show this, set $x \geq 3c$ and $y \geq 3c$. Thus, $K(9c) > K(3c) + K(3c) +
K(c)$. 

\section{Bonus Problem}

Provide a program that prints itself out.

See Exercise 6.1 for a listing of Clojure code that prints itself out when run.

\end{document}
